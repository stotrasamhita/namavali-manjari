% !TeX program = XeLaTeX
% !TeX root = ../nAmA.tex
\chapt{षण्मुखसहस्रनामावलिः}

\dnsub{ध्यानम्}
\twolineshloka
{ध्यायेत्षण्मुखमिन्दुकोटिसदृशं रत्नप्रभाशोभितम्}
{बालार्कद्युतिषट्किरीटविलसत्केयूरहारान्वितम्}

\twolineshloka
{कर्णालम्बितकुण्डलप्रविलसद्गण्डस्थलाशोभितम्}
{काञ्चीकङ्कणकिंकिणीरवयुतं श‍ृङ्गारसारोदयम्}

\twolineshloka
{ध्यायेदीप्सितसिद्धिदं शिवसुतं श्रीद्वादशाक्षं गुहम्}
{खेटं कुक्कुटमंकुशं च वरदं पाशं धनुश्चक्रकम्}

\twolineshloka
{वज्रं शक्तिमसिं च शूलमभयं दोर्भिर्धृतं षण्मुखम्}
{देवं चित्रमयूरवाहनगतं चित्राम्बरालंकृतम्}


\begin{multicols}{\maxColumns}
\begin{flushleft}
अचिन्त्यशक्तये~नमः\\
अनघाय~नमः\\
अक्षोभ्याय~नमः\\
अपराजिताय~नमः\\
अनाथवत्सलाय~नमः\\
अमोघाय~नमः\\
अशोकाय~नमः\\
अजराय~नमः\\
अभयाय~नमः\\
अत्युदाराय~नमः\hfill १०\\
अघहराय~नमः\\
अग्रगण्याय~नमः\\
अद्रिजासुताय~नमः\\
अनन्तमहिम्ने~नमः\\
अपाराय~नमः\\
अनन्तसौख्यप्रदाय~नमः\\
अव्ययाय~नमः\\
अनन्तमोक्षदाय~नमः\\
अनादये~नमः\\
अप्रमेयाय~नमः\hfill २०\\
अक्षराय~नमः\\
अच्युताय~नमः\\
अकल्मषाय~नमः\\
अभिरामाय~नमः\\
अग्रधुर्याय~नमः\\
अमितविक्रमाय~नमः\\
अनाथनाथाय~नमः\\
अमलाय~नमः\\
अप्रमत्ताय~नमः\\
अमरप्रभवे~नमः\hfill ३०\\
अरिन्दमाय~नमः\\
अखिलाधाराय~नमः\\
अणिमादिगुणाय~नमः\\
अग्रण्ये~नमः\\
अचञ्चलाय~नमः\\
अमरस्तुत्याय~नमः\\
अकलङ्काय~नमः\\
अमिताशनाय~नमः\\
अग्निभुवे~नमः\\
अनवद्याङ्गाय~नमः\hfill ४०\\
अद्भुताय~नमः\\
अभीष्टदायकाय~नमः\\
अतीन्द्रियाय~नमः\\
अप्रमेयात्मने~नमः\\
अदृश्याय~नमः\\
अव्यक्तलक्षणाय~नमः\\
आपद्विनाशकाय~नमः\\
आर्याय~नमः\\
आढ्याय~नमः\\
आगमसंस्तुताय~नमः\hfill ५०\\
आर्तसंरक्षणाय~नमः\\
आद्याय~नमः\\
आनन्दाय~नमः\\
आर्यसेविताय~नमः\\
आश्रितेष्टार्थवरदाय~नमः\\
आनन्दिने~नमः\\
आर्तफलप्रदाय~नमः\\
आश्चर्यरूपाय~नमः\\
आनन्दाय~नमः\\
आपन्नार्तिविनाशनाय~नमः\hfill ६०\\
इभवक्त्रानुजाय~नमः\\
इष्टाय~नमः\\
इभासुरहरात्मजाय~नमः\\
इतिहासश्रुतिस्तुत्याय~नमः\\
इन्द्रभोगफलप्रदाय~नमः\\
इष्टापूर्तफलप्राप्तये~नमः\\
इष्टेष्टवरदायकाय~नमः\\
इहामुत्रेष्टफलदाय~नमः\\
इष्टदाय~नमः\\
इन्द्रवन्दिताय~नमः\hfill ७०\\
ईडनीयाय~नमः\\
ईशपुत्राय~नमः\\
ईप्सितार्थप्रदायकाय~नमः\\
ईतिभीतिहराय~नमः\\
ईड्याय~नमः\\
ईषणात्र्यवर्जिताय~नमः\\
उदारकीर्तये~नमः\\
उद्योगिने~नमः\\
उत्कृष्टोरुपराक्रमाय~नमः\\
उत्कृष्टशक्तये~नमः\hfill ८०\\
उत्साहाय~नमः\\
उदाराय~नमः\\
उत्सवप्रियाय~नमः\\
उज्जृम्भाय~नमः\\
उद्भवाय~नमः\\
उग्राय~नमः\\
उदग्राय~नमः\\
उग्रलोचनाय~नमः\\
उन्मत्ताय~नमः\\
उग्रशमनाय~नमः\hfill ९०\\
उद्वेगघ्नोरगेश्वराय~नमः\\
उरुप्रभावाय~नमः\\
उदीर्णाय~नमः\\
उमापुत्राय~नमः\\
उदारधिये~नमः\\
ऊर्ध्वरेतःसुताय~नमः\\
ऊर्ध्वगतिदाय~नमः\\
ऊर्जपालकाय~नमः\\
ऊर्जिताय~नमः\\
ऊर्ध्वगाय~नमः\hfill १००\\
ऊर्ध्वाय~नमः\\
ऊर्ध्वलोकैकनायकाय~नमः\\
ऊर्जावते~नमः\\
ऊर्जितोदाराय~नमः\\
ऊर्जितोर्जितशासनाय~नमः\\
ऋषिदेवगणस्तुत्याय~नमः\\
ऋणत्र्यविमोचनाय~नमः\\
ऋजुरूपाय~नमः\\
ऋजुकराय~नमः\\
ऋजुमार्गप्रदर्शनाय~नमः\hfill ११०\\
ऋतम्बराय~नमः\\
ऋजुप्रीताय~नमः\\
ऋषभाय~नमः\\
ऋद्धिदाय~नमः\\
ऋताय~नमः\\
लुलितोद्धारकाय~नमः\\
लूतभवपाशप्रभञ्जनाय~नमः\\
एणाङ्कधरसत्पुत्राय~नमः\\
एकस्मै~नमः\\
एनोविनाशनाय~नमः\hfill १२०\\
ऐश्वर्यदाय~नमः\\
ऐन्द्रभोगिने~नमः\\
ऐतिह्याय~नमः\\
ऐन्द्रवन्दिताय~नमः\\
ओजस्विने~नमः\\
ओषधिस्थानाय~नमः\\
ओजोदाय~नमः\\
ओदनप्रदाय~नमः\\
औदार्यशीलाय~नमः\\
औमेयाय~नमः\hfill १३०\\
औग्राय~नमः\\
औन्नत्यदायकाय~नमः\\
औदार्याय~नमः\\
औषधकराय~नमः\\
औषधाय~नमः\\
औषधाकराय~नमः\\
अंशुमालिने~नमः\\
अंशुमालीड्याय~नमः\\
अम्बिकातनयाय~नमः\\
अन्नदाय~नमः\hfill १४०\\
अन्धकारिसुताय~नमः\\
अन्धत्वहारिणे~नमः\\
अम्बुजलोचनाय~नमः\\
अस्तमायाय~नमः\\
अमराधीशाय~नमः\\
अस्पष्टाय~नमः\\
अस्तोकपुण्यदाय~नमः\\
अस्तामित्राय~नमः\\
अस्तरूपाय~नमः\\
अस्खलत्सुगतिदायकाय~नमः\hfill १५०\\
कार्तिकेयाय~नमः\\
कामरूपाय~नमः\\
कुमाराय~नमः\\
क्रौञ्चदारणाय~नमः\\
कामदाय~नमः\\
कारणाय~नमः\\
काम्याय~नमः\\
कमनीयाय~नमः\\
कृपाकराय~नमः\\
काञ्चनाभाय~नमः\hfill १६०\\
कान्तियुक्ताय~नमः\\
कामिने~नमः\\
कामप्रदाय~नमः\\
कवये~नमः\\
कीर्तिकृते~नमः\\
कुक्कुटधराय~नमः\\
कूटस्थाय~नमः\\
कुवलेक्षणाय~नमः\\
कुङ्कुमाङ्गाय~नमः\\
क्लमहराय~नमः\hfill १७०\\
कुशलाय~नमः\\
कुक्कुटध्वजाय~नमः\\
कुशानुसम्भवाय~नमः\\
क्रूराय~नमः\\
क्रूरघ्नाय~नमः\\
कलितापहृते~नमः\\
कामरूपाय~नमः\\
कल्पतरवे~नमः\\
कान्ताय~नमः\\
कामितदायकाय~नमः\hfill १८०\\
कल्याणकृते~नमः\\
क्लेशनाशाय~नमः\\
कृपालवे~नमः\\
करुणाकराय~नमः\\
कलुषघ्नाय~नमः\\
क्रियाशक्तये~नमः\\
कठोराय~नमः\\
कवचिने~नमः\\
कृतिने~नमः\\
कोमलाङ्गाय~नमः\hfill १९०\\
कुशप्रीताय~नमः\\
कुत्सितघ्नाय~नमः\\
कलाधराय~नमः\\
ख्याताय~नमः\\
खेटधराय~नमः\\
खड्गिने~नमः\\
खट्वाङ्गिने~नमः\\
खलनिग्रहाय~नमः\\
ख्यातिप्रदाय~नमः\\
खेचरेशाय~नमः\hfill २००\\
ख्यातेहाय~नमः\\
खेचरस्तुताय~नमः\\
खरतापहराय~नमः\\
खस्थाय~नमः\\
खेचराय~नमः\\
खेचराश्रयाय~नमः\\
खण्डेन्दुमौलितनयाय~नमः\\
खेलाय~नमः\\
खेचरपालकाय~नमः\\
खस्थलाय~नमः\hfill २१०\\
खण्डितार्काय~नमः\\
खेचरीजनपूजिताय~नमः\\
गाङ्गेयाय~नमः\\
गिरिजापुत्राय~नमः\\
गणनाथानुजाय~नमः\\
गुहाय~नमः\\
गोप्त्रे~नमः\\
गीर्वाणसंसेव्याय~नमः\\
गुणातीताय~नमः\\
गुहाश्रयाय~नमः\hfill २२०\\
गतिप्रदाय~नमः\\
गुणनिधये~नमः\\
गम्भीराय~नमः\\
गिरिजात्मजाय~नमः\\
गूढरूपाय~नमः\\
गदहराय~नमः\\
गुणाधीशाय~नमः\\
गुणाग्रण्ये~नमः\\
गोधराय~नमः\\
गहनाय~नमः\hfill २३०\\
गुप्ताय~नमः\\
गर्वघ्नाय~नमः\\
गुणवर्धनाय~नमः\\
गुह्याय~नमः\\
गुणज्ञाय~नमः\\
गीतिज्ञाय~नमः\\
गतातङ्काय~नमः\\
गुणाश्रयाय~नमः\\
गद्यपद्यप्रियाय~नमः\\
गुण्याय~नमः\hfill २४०\\
गोस्तुताय~नमः\\
गगनेचराय~नमः\\
गणनीयचरित्राय~नमः\\
गतक्लेशाय~नमः\\
गुणार्णवाय~नमः\\
घूर्णिताक्षाय~नमः\\
घृणिनिधये~नमः\\
घनगम्भीरघोषणाय~नमः\\
घण्टानादप्रियाय~नमः\\
घोषाय~नमः\hfill २५०\\
घोराघौघविनाशनाय~नमः\\
घनानन्दाय~नमः\\
घर्महन्त्रे~नमः\\
घृणावते~नमः\\
घृष्टिपातकाय~नमः\\
घृणिने~नमः\\
घृणाकराय~नमः\\
घोराय~नमः\\
घोरदैत्यप्रहारकाय~नमः\\
घटितैश्वर्यसन्दोहाय~नमः\hfill २६०\\
घनार्थाय~नमः\\
घनसङ्क्रमाय~नमः\\
चित्रकृते~नमः\\
चित्रवर्णाय~नमः\\
चञ्चलाय~नमः\\
चपलद्युतये~नमः\\
चिन्मयाय~नमः\\
चित्स्वरूपाय~नमः\\
चिरानन्दाय~नमः\\
चिरन्तनाय~नमः\hfill २७०\\
चित्रकेलये~नमः\\
चित्रतराय~नमः\\
चिन्तनीयाय~नमः\\
चमत्कॄतये~नमः\\
चोरघ्नाय~नमः\\
चतुराय~नमः\\
चारवे~नमः\\
चामीकरविभूषणाय~नमः\\
चन्द्रार्ककोटिसदृशाय~नमः\\
चन्द्रमौलितनूभवाय~नमः\hfill २८०\\
चादिताङ्गाय~नमः\\
छद्महन्त्रे~नमः\\
छेदिताखिलपातकाय~नमः\\
छेदीकृततमःक्लेशाय~नमः\\
छत्रीकृतमहायशसे~नमः\\
छादिताशेषसन्तापाय~नमः\\
छरितामृतसागराय~नमः\\
छन्नत्रैगुण्यरूपाय~नमः\\
छातेहाय~नमः\\
छिन्नसंशयाय~नमः\hfill २९०\\
छन्दोमयाय~नमः\\
छन्दगामिने~नमः\\
छिन्नपाशाय~नमः\\
छविश्छदाय~नमः\\
जगद्धिताय~नमः\\
जगत्पूज्याय~नमः\\
जगज्ज्येष्ठाय~नमः\\
जगन्मयाय~नमः\\
जनकाय~नमः\\
जाह्नवीसूनवे~नमः\hfill ३००\\
जितामित्राय~नमः\\
जगद्गुरवे~नमः\\
जयिने~नमः\\
जितेन्द्रियाय~नमः\\
जैत्राय~नमः\\
जरामरणवर्जिताय~नमः\\
ज्योतिर्मयाय~नमः\\
जगन्नाथाय~नमः\\
जगज्जीवाय~नमः\\
जनाश्रयाय~नमः\hfill ३१०\\
जगत्सेव्याय~नमः\\
जगत्कर्त्रे~नमः\\
जगत्साक्षिणे~नमः\\
जगत्प्रियाय~नमः\\
जम्भारिवन्द्याय~नमः\\
जयदाय~नमः\\
जगज्जनमनोहराय~नमः\\
जगदानन्दजनकाय~नमः\\
जनजाड्यापहारकाय~नमः\\
जपाकुसुमसङ्काशाय~नमः\hfill ३२०\\
जनलोचनशोभनाय~नमः\\
जनेश्वराय~नमः\\
जितक्रोधाय~नमः\\
जनजन्मनिबर्हणाय~नमः\\
जयदाय~नमः\\
जन्तुतापघ्नाय~नमः\\
जितदैत्यमहाव्रजाय~नमः\\
जितमायाय~नमः\\
जितक्रोधाय~नमः\\
जितसङ्गाय~नमः\hfill ३३०\\
जनप्रियाय~नमः\\
झञ्जानिलमहावेगाय~नमः\\
झरिताशेषपातकाय~नमः\\
झर्झरीकृतदैत्यौघाय~नमः\\
झल्लरीवाद्यसम्प्रियाय~नमः\\
ज्ञानमूर्तये~नमः\\
ज्ञानगम्याय~नमः\\
ज्ञानिने~नमः\\
ज्ञानमहानिधये~नमः\\
टङ्कारनृत्तविभवाय~नमः\hfill ३४०\\
टङ्कवज्रध्वजाङ्किताय~नमः\\
टङ्किताखिललोकाय~नमः\\
टङ्कितैनस्तमोरवये~नमः\\
डम्बरप्रभवाय~नमः\\
डम्भाय~नमः\\
डम्बाय~नमः\\
डमरुकप्रियाय~नमः\\
डमरोत्कटसन्नादाय~नमः\\
डिम्बरूपस्वरूपकाय~नमः\\
ढक्कानादप्रीतिकराय~नमः\hfill ३५०\\
ढालितासुरसङ्कुलाय~नमः\\
ढौकितामरसन्दोहाय~नमः\\
ढुण्ढिविघ्नेश्वरानुजाय~नमः\\
तत्त्वज्ञाय~नमः\\
तत्त्वगाय~नमः\\
तीव्राय~नमः\\
तपोरूपाय~नमः\\
तपोमयाय~नमः\\
त्रयीमयाय~नमः\\
त्रिकालज्ञाय~नमः\hfill ३६०\\
त्रिमूर्तये~नमः\\
त्रिगुणात्मकाय~नमः\\
त्रिदशेशाय~नमः\\
तारकारये~नमः\\
तापघ्नाय~नमः\\
तापसप्रियाय~नमः\\
तुष्टिदाय~नमः\\
तुष्टिकृते~नमः\\
तीक्ष्णाय~नमः\\
तपोरूपाय~नमः\hfill ३७०\\
त्रिकालविदे~नमः\\
स्तोत्रे~नमः\\
स्तव्याय~नमः\\
स्तवप्रीताय~नमः\\
स्तुतये~नमः\\
स्तोत्राय~नमः\\
स्तुतिप्रियाय~नमः\\
स्थिताय~नमः\\
स्थायिने~नमः\\
स्थापकाय~नमः\hfill ३८०\\
स्थूलसूक्ष्मप्रदर्शकाय~नमः\\
स्थविष्ठाय~नमः\\
स्थविराय~नमः\\
स्थूलाय~नमः\\
स्थानदाय~नमः\\
स्थैर्यदाय~नमः\\
स्थिराय~नमः\\
दान्ताय~नमः\\
दयापराय~नमः\\
दात्रे~नमः\hfill ३९०\\
दुरितघ्नाय~नमः\\
दुरासदाय~नमः\\
दर्शनीयाय~नमः\\
दयासाराय~नमः\\
देवदेवाय~नमः\\
दयानिधये~नमः\\
दुराधर्षाय~नमः\\
दुर्विगाह्याय~नमः\\
दक्षाय~नमः\\
दर्पणशोभिताय~नमः\hfill ४००\\
दुर्धराय~नमः\\
दानशीलाय~नमः\\
द्वादशाक्षाय~नमः\\
द्विषड्भुजाय~नमः\\
द्विषट्कर्णाय~नमः\\
द्विषड्बाहवे~नमः\\
दीनसन्तापनाशनाय~नमः\\
दन्दशूकेश्वराय~नमः\\
देवाय~नमः\\
दिव्याय~नमः\hfill ४१०\\
दिव्याकृतये~नमः\\
दमाय~नमः\\
दीर्घवृत्ताय~नमः\\
दीर्घबाहवे~नमः\\
दीर्घदृष्टये~नमः\\
दिवस्पतये~नमः\\
दण्डाय~नमः\\
दमयित्रे~नमः\\
दर्पाय~नमः\\
देवसिंहाय~नमः\hfill ४२०\\
दृढव्रताय~नमः\\
दुर्लभाय~नमः\\
दुर्गमाय~नमः\\
दीप्ताय~नमः\\
दुष्प्रेक्ष्याय~नमः\\
दिव्यमण्डनाय~नमः\\
दुरोदरघ्नाय~नमः\\
दुःखघ्नाय~नमः\\
दुरारिघ्नाय~नमः\\
दिशाम्पतये~नमः\hfill ४३०\\
दुर्जयाय~नमः\\
देवसेनेशाय~नमः\\
दुर्ज्ञेयाय~नमः\\
दुरतिक्रमाय~नमः\\
दम्भाय~नमः\\
दृप्ताय~नमः\\
देवर्षये~नमः\\
दैवज्ञाय~नमः\\
दैवचिन्तकाय~नमः\\
धुरन्धराय~नमः\hfill ४४०\\
धर्मपराय~नमः\\
धनदाय~नमः\\
धृतवर्धनाय~नमः\\
धर्मेशाय~नमः\\
धर्मशास्त्रज्ञाय~नमः\\
धन्विने~नमः\\
धर्मपरायणाय~नमः\\
धनाध्यक्षाय~नमः\\
धनपतये~नमः\\
धृतिमते~नमः\hfill ४५०\\
धूतकिल्बिषाय~नमः\\
धर्महेतवे~नमः\\
धर्मशूराय~नमः\\
धर्मकृते~नमः\\
धर्मविदे~नमः\\
ध्रुवाय~नमः\\
धात्रे~नमः\\
धीमते~नमः\\
धर्मचारिणे~नमः\\
धन्याय~नमः\hfill ४६०\\
धुर्याय~नमः\\
धृतव्रताय~नमः\\
नित्यसत्त्वाय~नमः\\
नित्यतृप्ताय~नमः\\
निर्लेपाय~नमः\\
निश्चलात्मकाय~नमः\\
निरवद्याय~नमः\\
निराधाराय~नमः\\
निष्कलङ्काय~नमः\\
निरञ्जनाय~नमः\hfill ४७०\\
निर्ममाय~नमः\\
निरहङ्काराय~नमः\\
निर्मोहाय~नमः\\
निरुपद्रवाय~नमः\\
नित्यानन्दाय~नमः\\
निरातङ्काय~नमः\\
निष्प्रपञ्चाय~नमः\\
निरामयाय~नमः\\
निरवद्याय~नमः\\
निरीहाय~नमः\hfill ४८०\\
निर्दर्शाय~नमः\\
निर्मलात्मकाय~नमः\\
नित्यानन्दाय~नमः\\
निर्जरेशाय~नमः\\
निःसङ्गाय~नमः\\
निगमस्तुताय~नमः\\
निष्कण्टकाय~नमः\\
निरालम्बाय~नमः\\
निष्प्रत्यूहाय~नमः\\
निरुद्भवाय~नमः\hfill ४९०\\
नित्याय~नमः\\
नियतकल्याणाय~नमः\\
निर्विकल्पाय~नमः\\
निराश्रयाय~नमः\\
नेत्रे~नमः\\
निधये~नमः\\
नैकरूपाय~नमः\\
निराकाराय~नमः\\
नदीसुताय~नमः\\
पुलिन्दकन्यारमणाय~नमः\hfill ५००\\
पुरुजिते~नमः\\
परमप्रियाय~नमः\\
प्रत्यक्षमूर्तये~नमः\\
प्रत्यक्षाय~नमः\\
परेशाय~नमः\\
पूर्णपुण्यदाय~नमः\\
पुण्याकराय~नमः\\
पुण्यरूपाय~नमः\\
पुण्याय~नमः\\
पुण्यपरायणाय~नमः\hfill ५१०\\
पुण्योदयाय~नमः\\
परञ्ज्योतिषे~नमः\\
पुण्यकृते~नमः\\
पुण्यवर्धनाय~नमः\\
परानन्दाय~नमः\\
परतराय~नमः\\
पुण्यकीर्तये~नमः\\
पुरातनाय~नमः\\
प्रसन्नरूपाय~नमः\\
प्राणेशाय~नमः\hfill ५२०\\
पन्नगाय~नमः\\
पापनाशनाय~नमः\\
प्रणतार्तिहराय~नमः\\
पूर्णाय~नमः\\
पार्वतीनन्दनाय~नमः\\
प्रभवे~नमः\\
पूतात्मने~नमः\\
पुरुषाय~नमः\\
प्राणाय~नमः\\
प्रभवाय~नमः\hfill ५३०\\
पुरुषोत्तमाय~नमः\\
प्रसन्नाय~नमः\\
परमस्पष्टाय~नमः\\
पराय~नमः\\
परिवृढाय~नमः\\
पराय~नमः\\
परमात्मने~नमः\\
प्रब्रह्मणे~नमः\\
परार्थाय~नमः\\
प्रियदर्शनाय~नमः\hfill ५४०\\
पवित्राय~नमः\\
पुष्टिदाय~नमः\\
पूर्तये~नमः\\
पिङ्गलाय~नमः\\
पुष्टिवर्धनाय~नमः\\
पापहर्त्रे~नमः\\
पाशधराय~नमः\\
प्रमत्तासुरशिक्षकाय~नमः\\
पावनाय~नमः\\
पावकाय~नमः\hfill ५५०\\
पूज्याय~नमः\\
पूर्णानन्दाय~नमः\\
परात्पराय~नमः\\
पुष्कलाय~नमः\\
प्रवराय~नमः\\
पूर्वाय~नमः\\
पितृभक्ताय~नमः\\
पुरोगमाय~नमः\\
प्राणदाय~नमः\\
प्राणिजनकाय~नमः\hfill ५६०\\
प्रदिष्टाय~नमः\\
पावकोद्भवाय~नमः\\
परब्रह्मस्वरूपाय~नमः\\
परमैश्वर्यकारणाय~नमः\\
परर्धिदाय~नमः\\
पुष्टिकराय~नमः\\
प्रकाशात्मने~नमः\\
प्रतापवते~नमः\\
प्रज्ञापराय~नमः\\
प्रकृष्टार्थाय~नमः\hfill ५७०\\
पृथुवे~नमः\\
पृथुपराक्रमाय~नमः\\
फणीश्वराय~नमः\\
फणिवाराय~नमः\\
फणामणिविभुषणाय~नमः\\
फलदाय~नमः\\
फलहस्ताय~नमः\\
फुल्लाम्बुजविलोचनाय~नमः\\
फडुच्चाटितपापौघाय~नमः\\
फणिलोकविभूषणाय~नमः\hfill ५८०\\
बाहुलेयाय~नमः\\
बृहद्रूपाय~नमः\\
बलिष्ठाय~नमः\\
बलवते~नमः\\
बलिने~नमः\\
ब्रह्मेशविष्णुरूपाय~नमः\\
बुद्धाय~नमः\\
भुद्धिमतां वराय~नमः\\
बालरूपाय~नमः\\
ब्रह्मगर्भाय~नमः\hfill ५९०\\
ब्रह्मचारिणे~नमः\\
बुधप्रियाय~नमः\\
बहुश‍ृताय~नमः\\
बहुमताय~नमः\\
ब्रह्मण्याय~नमः\\
ब्राह्मणप्रियाय~नमः\\
बलप्रमथनाय~नमः\\
ब्रह्मणे~नमः\\
बहुरूपाय~नमः\\
बहुप्रदाय~नमः\hfill ६००\\
बृहद्भानुतनूद्भूताय~नमः\\
बृहत्सेनाय~नमः\\
बिलेशयाय~नमः\\
बहुबाहवे~नमः\\
बलश्रीमते~नमः\\
बहुदैत्यविनाशकाय~नमः\\
बिलद्वारान्तरालस्थाय~नमः\\
बृहच्छक्तिधनुर्धराय~नमः\\
बालार्कद्युतिमते~नमः\\
बालाय~नमः\hfill ६१०\\
बृहद्वक्षसे~नमः\\
बृहद्धनुषे~नमः\\
भव्याय~नमः\\
भोगीश्वराय~नमः\\
भाव्याय~नमः\\
भवनाशाय~नमः\\
भवप्रियाय~नमः\\
भक्तिगम्याय~नमः\\
भयहराय~नमः\\
भावज्ञाय~नमः\hfill ६२०\\
भक्तसुप्रियाय~नमः\\
भुक्तिमुक्तिप्रदाय~नमः\\
भोगिने~नमः\\
भगवते~नमः\\
भाग्यवर्धनाय~नमः\\
भ्राजिष्णवे~नमः\\
भावनाय~नमः\\
भर्त्रे~नमः\\
भीमाय~नमः\\
भीमपराक्रमाय~नमः\hfill ६३०\\
भूतिदाय~नमः\\
भूतिकृते~नमः\\
भोक्त्रे~नमः\\
भूतात्मने~नमः\\
भुवनेश्वराय~नमः\\
भावकाय~नमः\\
भीकराय~नमः\\
भीष्माय~नमः\\
भावकेष्टाय~नमः\\
भवोद्भवाय~नमः\hfill ६४०\\
भवतापप्रशमनाय~नमः\\
भोगवते~नमः\\
भूतभावनाय~नमः\\
भोज्यप्रदाय~नमः\\
भ्रान्तिनाशाय~नमः\\
भानुमते~नमः\\
भुवनाश्रयाय~नमः\\
भूरिभोगप्रदाय~नमः\\
भद्राय~नमः\\
भजनीयाय~नमः\hfill ६५०\\
भिषग्वराय~नमः\\
महासेनाय~नमः\\
महोदराय~नमः\\
महाशक्तये~नमः\\
महाद्युतये~नमः\\
महाबुद्धये~नमः\\
महावीर्याय~नमः\\
महोत्साहाय~नमः\\
महाबलाय~नमः\\
महाभोगिने~नमः\hfill ६६०\\
महामायिने~नमः\\
मेधाविने~नमः\\
मेखलिने~नमः\\
महते~नमः\\
मुनिस्तुताय~नमः\\
महामान्याय~नमः\\
महानन्दाय~नमः\\
महायशसे~नमः\\
महोर्जिताय~नमः\\
माननिधये~नमः\hfill ६७०\\
मनोरथफलप्रदाय~नमः\\
महोदयाय~नमः\\
महापुण्याय~नमः\\
महाबलपराक्रमाय~नमः\\
मानदाय~नमः\\
मतिदाय~नमः\\
मालिने~नमः\\
मुक्तामालाविभूषणाय~नमः\\
मनोहराय~नमः\\
महामुख्याय~नमः\hfill ६८०\\
महर्द्धये~नमः\\
मूर्तिमते~नमः\\
मुनये~नमः\\
महोत्तमाय~नमः\\
महोपाय~नमः\\
मोक्षदाय~नमः\\
मङ्गलप्रदाय~नमः\\
मुदाकराय~नमः\\
मुक्तिदात्रे~नमः\\
महाभोगाय~नमः\hfill ६९०\\
महोरगाय~नमः\\
यशस्कराय~नमः\\
योगयोनये~नमः\\
योगिष्ठाय~नमः\\
यमिनां वराय~नमः\\
यशस्विने~नमः\\
योगपुरुषाय~नमः\\
योग्याय~नमः\\
योगनिधये~नमः\\
यमिने~नमः\hfill ७००\\
यतिसेव्याय~नमः\\
योगयुक्ताय~नमः\\
योगविदे~नमः\\
योगसिद्धिदाय~नमः\\
यन्त्राय~नमः\\
यन्त्रिणे~नमः\\
यन्त्रज्ञाय~नमः\\
यन्त्रवते~नमः\\
यन्त्रवाहकाय~नमः\\
यातनारहिताय~नमः\hfill ७१०\\
योगिने~नमः\\
योगीशाय~नमः\\
योगिनां वराय~नमः\\
रमणीयाय~नमः\\
रम्यरूपाय~नमः\\
रसज्ञाय~नमः\\
रसभावनाय~नमः\\
रञ्जनाय~नमः\\
रञ्जिताय~नमः\\
रागिणे~नमः\hfill ७२०\\
रुचिराय~नमः\\
रुद्रसम्भवाय~नमः\\
रणप्रियाय~नमः\\
रणोदाराय~नमः\\
रागद्वेषविनाशनाय~नमः\\
रत्नार्चिषे~नमः\\
रुचिराय~नमः\\
रम्याय~नमः\\
रूपलावण्यविग्रहाय~नमः\\
रत्नाङ्गदधराय~नमः\hfill ७३०\\
रत्नभूषणाय~नमः\\
रमणीयकाय~नमः\\
रुचिकृते~नमः\\
रोचमानाय~नमः\\
रञ्जिताय~नमः\\
रोगनाशनाय~नमः\\
राजीवाक्षाय~नमः\\
राजराजाय~नमः\\
रक्तमाल्यानुलेपनाय~नमः\\
राजद्वेदागमस्तुत्याय~नमः\hfill ७४०\\
रजःसत्त्वगुणान्विताय~नमः\\
रजनीशकलारम्याय~नमः\\
रत्नकुण्डलमण्डिताय~नमः\\
रत्नसन्मौलिशोभाढ्याय~नमः\\
रणन्मञ्जीरभूषणाय~नमः\\
लोकैकनाथाय~नमः\\
लोकेशाय~नमः\\
ललिताय~नमः\\
लोकनायकाय~नमः\\
लोकरक्षाय~नमः\hfill ७५०\\
लोकशिक्षाय~नमः\\
लोकलोचनरञ्जिताय~नमः\\
लोकबन्धवे~नमः\\
लोकधात्रे~नमः\\
लोकत्रयमहाहिताय~नमः\\
लोकचूडामणये~नमः\\
लोकवन्द्याय~नमः\\
लावण्यविग्रहाय~नमः\\
लोकाध्यक्षाय~नमः\\
लीलावते~नमः\hfill ७६०\\
लोकोत्तरगुणान्विताय~नमः\\
वरिष्ठाय~नमः\\
वरदाय~नमः\\
वैद्याय~नमः\\
विशिष्टाय~नमः\\
विक्रमाय~नमः\\
विभवे~नमः\\
विबुधाग्रचराय~नमः\\
वश्याय~नमः\\
विकल्पपरिवर्जिताय~नमः\hfill ७७०\\
विपाशाय~नमः\\
विगतातङ्काय~नमः\\
विचित्राङ्गाय~नमः\\
विरोचनाय~नमः\\
विद्याधराय~नमः\\
विशुद्धात्मने~नमः\\
वेदाङ्गाय~नमः\\
विबुधप्रियाय~नमः\\
वचस्कराय~नमः\\
व्यापकाय~नमः\hfill ७८०\\
विज्ञानिने~नमः\\
विनयान्विताय~नमः\\
विद्वत्तमाय~नमः\\
विरोधिघ्नाय~नमः\\
वीराय~नमः\\
विगतरागवते~नमः\\
वीतभावाय~नमः\\
विनीतात्मने~नमः\\
वेदगर्भाय~नमः\\
वसुप्रदाय~नमः\hfill ७९०\\
विश्वदीप्तये~नमः\\
विशालाक्षाय~नमः\\
विजितात्मने~नमः\\
विभावनाय~नमः\\
वेदवेद्याय~नमः\\
विधेयात्मने~नमः\\
वीतदोषाय~नमः\\
वेदविदे~नमः\\
विश्वकर्मणे~नमः\\
वीतभयाय~नमः\hfill ८००\\
वागीशाय~नमः\\
वासवार्चिताय~नमः\\
वीरध्वंसाय~नमः\\
विश्वमूर्तये~नमः\\
विश्वरूपाय~नमः\\
वरासनाय~नमः\\
विशाखाय~नमः\\
विमलाय~नमः\\
वाग्मिने~नमः\\
विदुषे~नमः\hfill ८१०\\
वेदधराय~नमः\\
वटवे~नमः\\
वीरचूडामणये~नमः\\
वीराय~नमः\\
विद्येशाय~नमः\\
विबुधाश्रयाय~नमः\\
विजयिने~नमः\\
विनयिने~नमः\\
वेत्रे~नमः\\
वरीयसे~नमः\hfill ८२०\\
विरजासे~नमः\\
वसवे~नमः\\
वीरघ्नाय~नमः\\
विज्वराय~नमः\\
वेद्याय~नमः\\
वेगवते~नमः\\
वीर्यवते~नमः\\
वशिने~नमः\\
वरशीलाय~नमः\\
वरगुणाय~नमः\hfill ८३०\\
विशोकाय~नमः\\
वज्रधारकाय~नमः\\
शरजन्मने~नमः\\
शक्तिधराय~नमः\\
शत्रुघ्नाय~नमः\\
शिखिवाहनाय~नमः\\
श्रीमते~नमः\\
शिष्टाय~नमः\\
शुचये~नमः\\
शुद्धाय~नमः\hfill ८४०\\
शाश्वताय~नमः\\
श्रुतिसागराय~नमः\\
शरण्याय~नमः\\
शुभदाय~नमः\\
शर्मणे~नमः\\
शिष्टेष्टाय~नमः\\
शुभलक्षणाय~नमः\\
शान्ताय~नमः\\
शूलधराय~नमः\\
श्रेष्ठाय~नमः\hfill ८५०\\
शुद्धात्मने~नमः\\
शङ्कराय~नमः\\
शिवाय~नमः\\
शितिकण्ठात्मजाय~नमः\\
शूराय~नमः\\
शान्तिदाय~नमः\\
शोकनाशनाय~नमः\\
षाण्मातुराय~नमः\\
षण्मुखाय~नमः\\
षड्गुणैश्वर्यसंयुताय~नमः\hfill ८६०\\
षट्चक्रस्थाय~नमः\\
षडूर्मिघ्नाय~नमः\\
षडङ्गश्रुतिपारगाय~नमः\\
षड्भावरहिताय~नमः\\
षट्काय~नमः\\
षट्शास्त्रस्मृतिपारगाय~नमः\\
षड्वर्गदात्रे~नमः\\
षड्ग्रीवाय~नमः\\
षडरिघ्ने~नमः\\
षडाश्रयाय~नमः\hfill ८७०\\
षट्किरीटधराय श्रीमते~नमः\\
षडाधाराय~नमः\\
षट्क्रमाय~नमः\\
षट्कोणमध्यनिलयाय~नमः\\
षण्डत्वपरिहारकाय~नमः\\
सेनान्ये~नमः\\
सुभगाय~नमः\\
स्कन्दाय~नमः\\
सुरानन्दाय~नमः\\
सतां गतये~नमः\hfill ८८०\\
सुब्रह्मण्याय~नमः\\
सुराध्यक्षाय~नमः\\
सर्वज्ञाय~नमः\\
सर्वदाय~नमः\\
सुखिने~नमः\\
सुलभाय~नमः\\
सिद्धिदाय~नमः\\
सौम्याय~नमः\\
सिद्धेशाय~नमः\\
सिद्धिसाधनाय~नमः\hfill ८९०\\
सिद्धार्थाय~नमः\\
सिद्धसङ्कल्पाय~नमः\\
सिद्धसाधवे~नमः\\
सुरेश्वराय~नमः\\
सुभुजाय~नमः\\
सर्वदृशे~नमः\\
साक्षिणे~नमः\\
सुप्रसादाय~नमः\\
सनातनाय~नमः\\
सुधापतये~नमः\hfill ९००\\
स्वयम्ज्योतिषे~नमः\\
स्वयम्भुवे~नमः\\
सर्वतोमुखाय~नमः\\
समर्थाय~नमः\\
सत्कृतये~नमः\\
सूक्ष्माय~नमः\\
सुघोषाय~नमः\\
सुखदाय~नमः\\
सुहृदे~नमः\\
सुप्रसन्नाय~नमः\hfill ९१०\\
सुरश्रेष्ठाय~नमः\\
सुशीलाय~नमः\\
सत्यसाधकाय~नमः\\
सम्भाव्याय~नमः\\
सुमनसे~नमः\\
सेव्याय~नमः\\
सकलागमपारगाय~नमः\\
सुव्यक्ताय~नमः\\
सच्चिदानन्दाय~नमः\\
सुवीराय~नमः\hfill ९२०\\
सुजनाश्रयाय~नमः\\
सर्वलक्षण्सम्पन्नाय~नमः\\
सत्यधर्मपरायणाय~नमः\\
सर्वदेवमयाय~नमः\\
सत्याय~नमः\\
सदा मृष्टान्नदायकाय~नमः\\
सुधापिने~नमः\\
सुमतये~नमः\\
सत्याय~नमः\\
सर्वविघ्नविनाशनाय~नमः\hfill ९३०\\
सर्वदुःखप्रशमनाय~नमः\\
सुकुमाराय~नमः\\
सुलोचनाय~नमः\\
सुग्रीवाय~नमः\\
सुधृतये~नमः\\
साराय~नमः\\
सुराराध्याय~नमः\\
सुविक्रमाय~नमः\\
सुरारिघ्ने~नमः\\
स्वर्णवर्णाय~नमः\hfill ९४०\\
सर्पराजाय~नमः\\
सदाशुचये~नमः\\
सप्तार्चिर्भुवे~नमः\\
सुरवराय~नमः\\
सर्वायुधविशारदाय~नमः\\
हस्तिचर्माम्बरसुताय~नमः\\
हस्तिवाहनसेविताय~नमः\\
हस्तचित्रायुधधराय~नमः\\
हृताघाय~नमः\\
हसिताननाय~नमः\hfill ९५०\\
हेमभूषाय~नमः\\
हरिद्वर्णाय~नमः\\
हृष्टिदाय~नमः\\
हृष्टिवर्धनाय~नमः\\
हेमाद्रिभिदे~नमः\\
हंसरूपाय~नमः\\
हुङ्कारहतकिल्बिषाय~नमः\\
हिमाद्रिजातातनुजाय~नमः\\
हरिकेशाय~नमः\\
हिरण्मयाय~नमः\hfill ९६०\\
हृद्याय~नमः\\
हृष्टाय~नमः\\
हरिसखाय~नमः\\
हंसाय~नमः\\
हंसगतये~नमः\\
हविषे~नमः\\
हिरण्यवर्णाय~नमः\\
हितकृते~नमः\\
हर्षदाय~नमः\\
हेमभूषणाय~नमः\hfill ९७०\\
हरप्रियाय~नमः\\
हितकराय~नमः\\
हतपापाय~नमः\\
हरोद्भवाय~नमः\\
क्षेमदाय~नमः\\
क्षेमकृते~नमः\\
क्षेम्याय~नमः\\
क्षेत्रज्ञाय~नमः\\
क्षामवर्जिताय~नमः\\
क्षेत्रपालाय~नमः\hfill ९८०\\
क्षमाधाराय~नमः\\
क्षेमक्षेत्राय~नमः\\
क्षमाकराय~नमः\\
क्षुद्रघ्नाय~नमः\\
क्षान्तिदाय~नमः\\
क्षेमाय~नमः\\
क्षितिभूषाय~नमः\\
क्षमाश्रयाय~नमः\\
क्षालिताघाय~नमः\\
क्षितिधराय~नमः\hfill ९९०\\
क्षीणसंरक्षणक्षमाय~नमः\\
क्षणभङ्गुरसन्नद्धघनशोभिकपर्दकाय~नमः\\
क्षितिभृन्नाथतनयामुखपङ्कजभास्कराय~नमः\\
क्षताहिताय~नमः\\
क्षराय~नमः\\
क्षन्त्रे~नमः\\
क्षतदोषाय~नमः\\
क्षमानिधये~नमः\\
क्षपिताखिलसन्तापाय~नमः\\
क्षपानाथसमाननाय~नमः\hfill १०००
\end{flushleft}
\end{multicols}
॥इति श्रीस्कान्दे महापुराणे ईश्वरप्रोक्ते ब्रह्मनारदसंवादे षण्मुखसहस्रनामावलिः सम्पूर्णा॥

\dnsub{फलश्रुतिः}
\twolineshloka
{इति नाम्नां सहस्राणि षण्मुखस्य च नारद}
{यः पठेच्छृणुयाद्वापि भक्तियुक्तेन चेतसा}

\twolineshloka
{स सद्यो मुच्यते पापैर्मनोवाक्कायसम्भवैः}
{आयुर्वृद्धिकरं पुंसां स्थैर्यवीर्यविवर्धनम्}

\twolineshloka
{वाक्येनैकेन वक्ष्यामि वाञ्छितार्थं प्रयच्छति}
{तस्मात्सर्वात्मना ब्रह्मन्नियमेन जपेत्सुधीः}


